\begin{titlepage}
    \section{Introduction to Master's Thesis}
    Humans depend on well-functioning ecosystems for survival and have
    recognized as such in a number of nature conservation laws and treaties such
    as the Convention on Biological Diversity. The European Union has
    implemented the Habitats Directive (Council Directive) 92/43/EEC [1992] to
    comply with the CBD and protect bioidiversity in the EU. As part of the
    directive, periodic reports detailing status need to be submitted every six
    years. The true potential of the directive is not yet realized as the data
    in the reports is difficult to compare due to varying data collection
    methods and aquistition nomenclatures. The differing collection methods of
    member states when performing field surveys, which are already subjective,
    compounds the problem (Cherrill 199). Luckily, with remote sensing data,
    Geographic Object Based Image Analysis (GEOBIA) and ontologies stored as
    OWL2/XML files, these biodiversity reports could become cheaper, more
    objective and less time consuming.

    This Master's Thesis presents a method for a semi-automatic ontology-based
    classification system that has been applied to formalized EUNIS dry, mesic,
    wet (E1, E2, E3) grasslands in Saarburg, Rhineland-Palatinate. I have chosen
    to write this thesis in the form of a scientific journal article to
    disseminate my findings in the hopes that it might be useful to someone
    else.

    \section{Acknowledgements}
    This thesis would not be possible without the help of Simon Nieland and the
    NATFLO project. I thank Simon for all his patience, guidance and technical
    support. I would like to thank Professor Dr. Birgit Kleinschmit for her feedback
    and the opportunity to work in her lab; as without it this thesis would not
    exist. I would like to thank all of my colleagues for a great work
    environment and collaboration. 

    %The field of Environmental Planning has gone through a renaissance due to
    %the availability of free global remote sensing products, ``open data'' and
    %inexpensive commodity hardware that makes processing and analysing the data
    %much easier. The availability of free/libre open soure machine learning
    %software and other tools to process  earth observation data has brought
    %great changes to the field but also great challenges. One such challenge is
    %in data and workflow management.
    %The need for reproducea\cite{Arvor2013}

    
\end{titlepage}
