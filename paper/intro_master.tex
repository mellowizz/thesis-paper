\pagebreak
\begin{center}{\textbf{Introduction to Master's Thesis}}\end{center}
Humans depend on well-functioning ecosystems for survival and have
recognized as such in a number of nature conservation laws and treaties such
as the Convention on Biological Diversity(CBD). The European Union has
implemented the Habitats Directive (Council Directive) 92/43/EEC [1992] to
comply with the CBD and protect biodiversity in the EU. As part of the
directive, periodic reports detailing status need to be submitted every six
years. The true potential of the directive is not yet realized as the data
in the reports is difficult to compare due to varying data collection
methods and aquisition nomenclatures. The differing collection methods of
member states when performing field surveys, which are already subjective,
compounds the problem (Cherrill 199). Luckily, with remote sensing data,
Geographic Object Based Image Analysis (GEOBIA) and ontologies stored as
OWL2/XML files, these biodiversity reports could become cheaper, more
objective and less time consuming.

This Master's Thesis presents a method for a semi-automatic ontology-based
classification system that has been applied to formalized EUNIS dry, mesic,
wet (E1, E2, E3) grasslands in Saarburg, Rhineland-Palatinate. I have chosen
to write this thesis in the form of a scientific journal article to
disseminate my findings in the hopes that it might be useful to someone
else. In the pages that follow is the soon to be submitted journal article. At 
the end of the article are links to the OWL ontologies.

\begin{center}{\textbf{Zusaammenfassung}}\end{center}
Die Fernerkundung ist entscheidend f\"ur das Biodiversit\"at-Monitoring. Sie 
hilft die Anforderungen von bestehenden Gesetzen, wie z.B. der 
EU-Habitatrichtlinie, zu erf\"ullen. Voll eingesetzt bringt sie die 
M\"oglichkeit, in der Zukunft eine Vielfalt von aufkommenden Herausforderungen 
zu \"uberwinden und etwaige Probleme zu l\"osen. Trotz dieser 
positiven Attribute ist das volle Potenzial der Fernerkundung jedoch noch nicht 
realisiert, weil verschiedene Begriffe und Klassifikationssystem in 
Biodiversit\"at-Monitoring gewendet wird. Ergebnisse zu Vergleichen ist nicht 
einfach, Daten-austauschen gleich so und Fragen von Herkunft/Herangehensweisen 
sind offen. Au\sserdem sind Feldmesskampagnen subjektiv, teuer und 
arbeitsaufwendig. Automatisierten Software und Methoden die Nachvollziehbare, 
Vergleichbare und Objektive Ergebnisse liefern ist n\"otig. Diese Probleme sind 
in einen Semi-Automatisierten Ontologie-basierte-Klassifikation die Data 
Mining Algorithmen verwendete um empirische Klassifikationergebnisse auszugeben. 
Die Methode wurde in Saarburg, Rheinland Pfalz an EUNIS Grasland Biotopen 
verwendet und einen ersten schritt um einen Automatisierten 
Biodiversit\"ats-Monitoring System zu realisieren. 

\begin{center}{\textbf{Acknowledgements}}\end{center}
This thesis would not be possible without the help of Simon Nieland and the 
NATFLO project. I thank Simon for all his patience, guidance and for leading me 
to this great field of research. I would like to thank Professor 
Dr. Birgit Kleinschmit for her feedback and the opportunity to work in her lab; 
as without it this thesis would not exist. I would like to thank all of my 
colleagues for a great work environment and rewarding collaboration.