\documentclass[authoryear,preprint,12pt,number]{elsarticle}
%\documentclass[authoryear,review,12pt,number]{elsarticle}
\usepackage[utf8]{inputenc}
\usepackage[T1]{fontenc}
\usepackage[numbers]{natbib}
\usepackage{graphicx}
\usepackage{float}
\usepackage{rotating}
\usepackage{stfloats}
%\usepackage{lineno}
%\usepackage[linesnumbered,ruled,vlined]{algorithm2e}
\usepackage{tabulary}
\usepackage{graphicx}
\usepackage[none]{hyphenat}
%\usepackage[table]{xcolor} \sloppy
\usepackage[hyphens]{url}
\usepackage{hyperref}
\usepackage[toc]{glossaries}
\usepackage{amsmath}
\usepackage{multirow}
\usepackage{rotating}
\usepackage{adjustbox}
\usepackage{graphicx}% http://ctan.org/pkg/graphicx
\usepackage{booktabs}% http://ctan.org/pkg/booktabs
\usepackage{xparse}% http://ctan.org/pkg/xparse
\usepackage{booktabs}
\usepackage{array}
\usepackage[ddmmyyyy]{datetime}
\usepackage{listings}
\usepackage{color}
\usepackage{standalone}
\usepackage[final]{pdfpages}

\renewcommand{\dateseparator}{.}

\newcolumntype{R}[2]{%
    >{\adjustbox{angle=#1,lap=\width-(#2)}\bgroup}%
    l%
    <{\egroup}%
}
\newcommand*\rot{\multicolumn{1}{R{60}{1em}}}% no optional argument here,
% please!

\loadglsentries[main]{./glossary}
\makeglossaries
\begin{document}
\begin{titlepage}
\begin{center}

% Upper part of the page. The '~' is needed because \\
% only works if a paragraph has started.
\textsc{\LARGE Technische Universit\"at Berlin}\\[0.5cm]
\textsc{Master Thesis}\\[1.5cm]

\textsc{\Large Classifying EUNIS Habitats using Ontologies and Data Mining
Methods}\\[1.5cm]
\textsc{\textit{A thesis submitted in partial fulfillment of the requirements
for the degree of}}\\[1.25cm]
\textsc{\Large Master of Science in Environmental Planning}\\[1.5cm]
\textsc{Faculty VI - Planning Building Environment\\
 Geoinformation in Environmental Planning}\\[1.5cm]

% Title
% Author and supervisor
\noindent
\begin{minipage}{0.5\textwidth}
\begin{flushleft} \large
\emph{Author:}\\
T. Niklas \textsc{Moran}
\end{flushleft}
\end{minipage}%
\begin{minipage}{0.5\textwidth}
\begin{flushright} \large
\emph{Supervisor:} \\
Prof. Dr.~Birgit \textsc{Kleinschmit}
Mag. rer. nat.~Simon \textsc{Nieland}
\end{flushright}
\end{minipage}

\vfill

% Bottom of the page
{\large January 2015}

\end{center}
\end{titlepage}

\includepdf[pages=-]{./authordec.pdf}
%\begin{titlepage}
\begin{center}
{Author's Declaration}
\end{center}
I hereby certify that I am the sole author of this master thesis. 
Furthermore, I confirm that no sources have been used in the preparation of this thesis other
than those indicated in the thesis itself. The works of other people included in
my thesis, published or otherwise, are fully acknowledged in accordance with the
standard referencing practices. This thesis has not been submitted for another
degree or master to any other University or Institution.
\\
\\
Hiermit versichere
ich, dass ich die vorliegende Arbeit selbstst�ndig verfasst und keine anderen als die
angegebenen Quellen und Hilfsmittel benutzt habe. Alle Ausf�hrungen, die anderen
ver�ffentlichten oder nicht ver�ffentlichten Schriften w�rtlich oder
sinngem\"a\ss entnommen wurden, habe ich kenntlich gemacht. Die Arbeit hat in
gleicher oder \"ahnlicher Fassung noch keiner anderen Pr\"ufungsbeh\"orde vorgelegen.

\vfill
\begin{center}
\noindent
\begin{minipage}{0.5\textwidth}
\begin{flushleft}
\rule{5cm}{0.4pt}
Date/Datum 
\end{flushleft}
\end{minipage}%
\begin{minipage}{0.5\textwidth}
\begin{flushright} 
\rule{5cm}{0.4pt}
Signature/Unterschrift
\end{flushright}
\end{minipage}


\end{center}
\end{titlepage}
\tableofcontents
\listoffigures
\listoftables
\printglossary
\pagebreak
\begin{center}{\textbf{Introduction to Master's Thesis}}\end{center}
Humans depend on well-functioning ecosystems for survival and have
recognized as such in a number of nature conservation laws and treaties such
as the Convention on Biological Diversity(CBD). The European Union has
implemented the Habitats Directive (Council Directive) 92/43/EEC [1992] to
comply with the CBD and protect biodiversity in the EU. As part of the
directive, periodic reports detailing status need to be submitted every six
years. The true potential of the directive is not yet realized as the data
in the reports is difficult to compare due to varying data collection
methods and aquisition nomenclatures. The differing collection methods of
member states when performing field surveys, which are already subjective,
compounds the problem (Cherrill 199). Luckily, with remote sensing data,
Geographic Object Based Image Analysis (GEOBIA) and ontologies stored as
OWL2/XML files, these biodiversity reports could become cheaper, more
objective and less time consuming.

This Master's Thesis presents a method for a semi-automatic ontology-based
classification system that has been applied to formalized EUNIS dry, mesic,
wet (E1, E2, E3) grasslands in Saarburg, Rhineland-Palatinate. I have chosen
to write this thesis in the form of a scientific journal article to
disseminate my findings in the hopes that it might be useful to someone
else. In the pages that follow is the soon to be submitted journal article. At 
the end of the article are links to the OWL ontologies.

\begin{center}{\textbf{Zusaammenfassung}}\end{center}
Die Fernerkundung ist entscheidend f\"ur das Biodiversit\"at-Monitoring. Sie 
hilft die Anforderungen von bestehenden Gesetzen, wie z.B. der 
EU-Habitatrichtlinie, zu erf\"ullen. Voll eingesetzt bringt sie die 
M\"oglichkeit, in der Zukunft eine Vielfalt von aufkommenden Herausforderungen 
zu \"uberwinden und etwaige Probleme zu l\"osen. Trotz dieser 
positiven Attribute ist das volle Potenzial der Fernerkundung jedoch noch nicht 
realisiert, weil verschiedene Begriffe und Klassifikationssystem in 
Biodiversit\"at-Monitoring gewendet wird. Ergebnisse zu Vergleichen ist nicht 
einfach, Daten-austauschen gleich so und Fragen von Herkunft/Herangehensweisen 
sind offen. Au\sserdem sind Feldmesskampagnen subjektiv, teuer und 
arbeitsaufwendig. Automatisierten Software und Methoden die Nachvollziehbare, 
Vergleichbare und Objektive Ergebnisse liefern ist n\"otig. Diese Probleme sind 
in einen Semi-Automatisierten Ontologie-basierte-Klassifikation die Data 
Mining Algorithmen verwendete um empirische Klassifikationergebnisse auszugeben. 
Die Methode wurde in Saarburg, Rheinland Pfalz an EUNIS Grasland Biotopen 
verwendet und einen ersten schritt um einen Automatisierten 
Biodiversit\"ats-Monitoring System zu realisieren. 

\begin{center}{\textbf{Acknowledgements}}\end{center}
This thesis would not be possible without the help of Simon Nieland and the 
NATFLO project. I thank Simon for all his patience, guidance and for leading me 
to this great field of research. I would like to thank Professor 
Dr. Birgit Kleinschmit for her feedback and the opportunity to work in her lab; 
as without it this thesis would not exist. I would like to thank all of my 
colleagues for a great work environment and rewarding collaboration.

\begin{frontmatter}
%\linenumbers
\title{Classifying EUNIS habitats using ontologies and data mining 
methods}

\author[TUB]{T. Niklas Moran\corref{cor1}}
\ead{niklasmoran@mailbox.tu-berlin.de}

\author[TUB]{Simon Nieland}
\author[TUB]{Birgit Kleinschmit}

\address[TUB]{Geoinformation in Environmental Planning Lab, Technische
Universit\"at Berlin, Stra\ss e des 17. Juni 145, 10623 Berlin, Germany}

\cortext[cor1]{Corresponding author at: Geoinformation in Environmental Planning
Lab, Technische Universit\"at Berlin, Stra\ss e des 17. Juni 145, 10623 Berlin,
Germany}

\begin{abstract}
Biodiversity monitoring using Remote Sensing is critical to meet 
requirements of existing laws such as the EU Habitats Directive and more 
importantly meet future challenges. The full potential of RS has yet 
to be harnessed as different nomenclatures and procedures hinder 
interoperability, comparison and provenance. Furthermore, the manual field 
surveys are expensive and time-consuming. To meet 
these future challenges automated tools are needed to use RS data to 
produce comparable, empirical data outputs that lend 
themselves to data discovery and provenance. These issues are addressed by a 
novel semi-automatic ontology-based classification method that uses data mining 
algorithms and OWL ontologies that yields traceable, interoperable and 
observation-based classification outputs. The method is tested on EUNIS 
grasslands in Saarburg, Rheinland-Palatinate. The method is a first 
step in developing observation-based ontologies in the field of nature 
conservation. 
especially for data discovery, automatic image interpretation, data 
interoperability, workflow management and data publication.
\end{abstract}

\begin{keyword}
remote sensing, biotope classification, data mining, nature conservation, OWL, 
EUNIS, GEOBIA 
\end{keyword}
\end{frontmatter}
%\linenumbers
\section{Introduction}
Recognizing the importance of functioning ecosystems to reduce biodiversity 
loss, the European Union has implemented an environmental conservation 
framework to protect and conserve vital habitats in accordance with the 
Convention on Biological Diversity. An integral part of this framework is the 
EU Habitats Directive (Council Directive) 92/43/EEC [1992], which established 
the Natura 2000 network of habitats. The directive requires member states to 
conserve and monitor designated habitats and for a report to be submitted every 
six years. Environmental data to determine biodiversity status must be 
collected to comply with reporting requirements. Yet, comparing data used for 
these reports is difficult due to varying data collection methods and 
acquisition nomenclatures by the nature conservation authorities in each member 
state ~\citep{VandenBorre2011}. The main issue lies in the subjective nature of 
field surveys to identify habitats ~\citep{Cherrill1999, Cherrill1999a, 
Hearn2011, Nieland2015a}. Furthermore, habitat status is mostly generated in 
bottom-up approaches taking into account the national and regional 
interpretation guidelines ~\citep{VandenBorre2011, INSPIREdataspecs}. This 
subjective and time-consuming task of conducting field surveys could be 
partially replaced with an automated \gls{rs} method that uses \gls{geobia} to 
reduce subjectivity, costs and time.

Remote sensing offers opportunities to collect and automatically interpret 
large amounts of computer-readable data useful for nature conservation and 
biodiversity monitoring ~\citep{Corbane2015, VandenBorre2011, Mayer2011}. 
\gls{rs} image analysis implicitly incorporates the expertise of the person 
performing the analysis, reducing reproducibility as the analyst ultimately 
chooses class membership in non-crisp boundaries between classes. This can be 
divided into remote sensing knowledge (spectral signature, remote sensing 
indices, etc.) and field knowledge (feature properties, spatial relations, 
etc.) ~\citep{Andres2013a}, which is often neither completely nor explicitly 
defined as it is based on trial and error but influences the classification 
~\citep{Arvor2013}. To ensure accuracy and applicability of classification 
outputs for conservation, experts with detailed knowledge of the sites are 
needed to interpret the \gls{rs} data. The distance between the high-level 
semantics used by experts to describe domain concepts and the low-level 
information quantified from data is referred to as the ``semantic 
gap''~\citep{Smeulders00}.

Ontologies can help bridge the ``semantic gap'' and allow for better data 
transferability, knowledge and workflow management (provenance) and logical 
consistency ~\citep{Janowicz2012}.  The standards-compliant format designed and 
adopted to express rich semantics and enable the ``Semantic Web'' is called the 
(\gls{owl})\footnote{\url{http://www.w3.org/TR/owl2-overview/}}. The format 
supports multiple syntaxes yet defines the \gls{rdf} (subject, 
predicate, object triplets) saved as \gls{xml} as a common exchange format.  
Moreover, through the use of reasoners (inference engines) that infer logical 
consequences over axioms and asserted facts and verify consistency, one can 
discover new knowledge ~\citep{Arvor2013, Andres2013a} \gls{rs} and field expert 
knowledge can be digitized in ontologies, thus allowing for a hierarchy of 
concepts for improved automatic image annotation and retrieval using concepts 
from both fields to produce more accurate results ~\citep{Srikanth2005}. 
~\cite{Janowicz2012} advocates for more observation-driven ontologies and for 
including machine learning, statistics and data mining to construct ontological 
primitives. While published research on using observation-based ontologies for 
biotope classifications is limited, the available research using ontologies in 
\gls{rs} research is briefly summarized below.

Ontologies modeled on the Land Cover Classification System and the General 
Habitat Category were integrated into tools used to monitor and protect areas 
in the EU ~\citep{Arvor2013}. The authors note that using the taxonomy of the 
different classification systems makes it possible to include expert knowledge 
in the process. ~\cite{Lucas2015} used pixel-based analysis and \gls{geobia} 
for greater classification accuracy which relies on a rule-base created by an 
expert. Other research includes classification of urban building types using a 
three-layered architecture ~\citep{diSciascio2013} and a semi-automated 
classification of urban building using the Random Forest classifier to 
determine variable importance of features from airborne laser scanner data 
~\citep{Belgiu2014}. Ontologies have also been paired with different algorithms 
to automatically acquire classification rules: a genetic programming algorithm 
~\citep{Forestier2012470} and the C4.5 data mining algorithm 
~\citep{Sheeren2006ML}. In biodiversity monitoring research, ontologies have 
been demonstrated to improve spatial data interoperability 
~\citep{Nieland2015a, Nieland2015b} and have been shown to aid in discovery of 
new relationships to consider for habitat management ~\citep{Perez-Luque2015}. 
The addition of fuzzy data types to \gls{owl} and the development of a fuzzy 
spatial reasoner holds great promise for the future of \gls{geobia} ontology 
research using remote sensing ~\citep{Belgiu2013, Bobillo2015}. More recently a 
multi-scale fuzzy spatial reasoner was developed which could have significant 
impact on this research ~\citep{Argyridis2015}.

Even though researchers recently developed a number of indicators using 
different sensors for habitat evaluation ~\citep{Nagendra2013}, classification 
procedures and rule-sets were not formalized to be computer readable and 
therefore suffer from similar transferablity and reproducibility problems as 
manual habitat mapping ~\citep{Arvor2013, Nieland2015a, Nieland2015b}. 
Therefore a formalized computer-readable ontology could help solve these 
problems and allow scientists to see how the classification was performed and 
be aware of possible incompatibilities before combining data 
~\citep{Janowicz2012}. Furthermore, there is no standardized set of indicators 
using \gls{rs} for trans-national habitat evaluation ~\citep{Lucas2015, 
VandenBorre2011}. Therefore, technical solutions to increase interoperability 
by thematically harmonizing environmental data and systematize data collection 
methods from remote sensing inputs in an automated workflow are needed. 

The \gls{eagle} is an expert group that seeks to harmonize \gls{lc}
and \gls{lu} nomenclatures using an object-oriented data model that 
eases translations between nomenclatures ~\citep{arnold2013eagle}. The many 
different nomenclatures used in Europe each have their own specific thematic 
conceptualization suited towards a specific scale and data collection method- 
reducing the ability to compare thematic maps. Since LU and LC are 
interconnected and influence one another, nomenclatures often incorporate both 
definitions into one class making separation difficult. To overcome this 
problem the \gls{eagle} data model describes landscapes in three main 
components: land cover (abiotic, vegetation, water) land use (agriculture, 
forestry, etc.) and characteristics (bio-physical, cultivation etc). The 
increased interoperability and transferability of \gls{rs} data and the 
semantic layer on top helps decision-makers to better assess and compare 
outcomes. 


In this paper we propose an automated system that can classify dry, mesic and 
wet grassland habitats according to the \gls{eunis} biotope classification 
schema using earth observation data, existing thematic maps (biotope, forestry, 
etc.), and expert knowledge formalized in an ontology by taking into account 
rules generated by data mining algorithms. The combination of data mining 
algorithms with ontology-based classification has, to our knowledge, not yet 
been done and is a first in remote sensing research. This method contributes to 
the goal of empirically-derived rule creation and enhances data 
interoperability and comparison as proposed ~\cite{Janowicz2012}. The main 
goals of this paper are:
\begin{itemize}
 \item to develop a \gls{rs} classification methodology using data mining 
approaches
     in combination with ontological formalism to generate highly interoperable,
     reproducable and exchangeable classification procedures and results,
 \item apply the methodology to indicators used to separate grassland habitats
     defined under \gls{eunis}
 \item and evaluate the developed approach by comparing it to a state-of-the art
     machine learning classifier \gls{et}
\footnote{\url{
http://scikit-learn.org/stable/modules/ensemble.html\#extremely-randomized-t
rees}}.
\end{itemize}
\section{Method}
\subsection{Use case: Classifying EUNIS Grasslands in 
Rhineland-Palatinate}
\label{sec:usecase_data}
To test the feasibility of applying our method to a real-world use case, we 
tested our method on dry, mesic and wet grasslands (\gls{eunis} E1, E2 and E3 
respectively) in the district of Saarburg. 

Saarburg is an 200km$^{2}$ administrative district and is located in the
south-west of the federal state of \gls{rlp}, Germany.
Luxembourg borders the area to the west and the federal state of Saarland to
the South. \gls{rlp} has a western european atlantic climate and has an 
economically
and culturally important viticulture industry along the Mosel and Rhine rivers
(see figure \ref{fig:study_area}).
\begin{figure}
    \includegraphics[width=\textwidth]{diagrams/study_area_closeup.png}
    \caption{The location of Saarburg (in purple) in relation to
    Rheinland-Palatinate. Map on right \copyright Thunderforest, Data\copyright
    OpenStreetMap contributors.}
\label{fig:study_area}
\end{figure}
To achieve a accurate formalization of the regarded habitats we adapted 
selected indicators, defined in the \gls{eunis} nomenclature, to meet the 
requirements of remote sensing analysis. Furthermore, we adopted \gls{eagle}'s 
object-oriented approach by separating land use, land cover and 
characteristics (biophysical and anthropomorphic) and adopted terms when 
possible to increase interoperability and further re-use. The table 
~\ref{tab:indicators_classes} shows the list of indicators that can be detected 
by our data and aggregated to form \gls{eunis} classes. 
\begin{table}
\centering
  \begin{tabular}{clcccccccc}
  \rot{description}&\rot{\gls{eunis} class}&\rot{wetness} & \rot{vegetation 
type} &
  \rot{usage} & \rot{usage intensity} & \rot{immature
  soil} & \rot{hydromorphic} & \rot{species richness} \\ \hline
\multirow{2}{*}{dry}
    & E1   & dry & g/h & g/m/- & low & 1 & 0 & 0/1/- \\ 
    & E1.2 & dry & g/h & g/m/- & low & 1 & 0 & 1\\
\multirow{3}{*}{mesic} 
    & E2   & mesic & g/h & g/m/- & l/m/h/- & 0 & 0 & 0/1/-\\
    & E2.1 & mesic & g/h & g & medium/high & 0 & 0 & 0/1/- \\
    & E2.6 & mesic & g & g/m/- & high & 0 & 0 & 0 \\
\multirow{2}{*}{wet}
    & E3   & very wet & g/h & g/m/- & low/medium & 0 & 1 & 0/1/- \\
    & E3.4 & very wet & g/h & g/m & medium & 0 & 1 & 0/1 \\
\end{tabular}
\caption{A `/ ' denotes OR and a '-' denotes 'none' and the first letter of 
each value is used to safe space. Vegetation type: \{graminaceous, 
herbaceous\}, usage: \{grazing, mowing\}, usage intensity: \{low, medium, 
high\}.\label{tab:indicators_classes}}
\end{table}
All described processes are based on a pre-segmentation based on \gls{geobia}.  
For this pre-segmentation step two types of data were taken into account.  
Digital orthophotos (Year:2012-2013, resolution 0.2m) are used to create the 
Digital Surface Model (\gls{dsm}) (resolution 0.5m) using automated stereo 
matching and therefore represent the basis for all segmentation and 
classification procedures. The \gls{dtm} and \gls{dem} was produced using 
\gls{lidar} ASCII point clouds acquired between 2003 and 2009 with a 
resolution of 0.2m.

The reference data (see 
\ref{subsec:reference_data_and_semantic_characterisation}) consists of the 
federal biotope map (partially updated 2015), agricultural data updated every 
year and a soil map which is based on the ALKIS (Amtliches Liegenschaftskaster 
Informationssystem- Automated Land Registration Map) (partially updated 2015). 
The actual classification process is based on two RapidEye Scenes from 2014 and 
indices of the \gls{dtm}, \gls{dem} and orthophotos.

\begin{figure} \includegraphics[width=1\textwidth]{diagrams/pre_processing.png}
    \caption{biotope, agriculture and soil maps are combined 
    (union) and joined with the segmented polygons (derived from 
    orthophotos) fitting within the combined polygons. The features are joined 
    together and all properties associated from the data are appended unless 
    they conflict. When conflicts occur the corresponding column is marked as 
    such.\label{fig:pre-processing}}
\end{figure}

This section describes the developed ontology-based, multi-source \gls{eunis} 
habitat classification method.
\label{subsec:method_overview}
\subsection{Overview of the automated EUNIS Habitat Mapping System}
Figure ~\ref{fig:full_workflow} gives an overview of the method using the
wetness indicator as an example of one of the habitat indicators needed to
differentiate dry, mesic and wet grassland habitats according to level 2 of
\gls{eunis}. The developed system is comprised of (1) the preparation of 
reference
data, including a spatial union of the base data sets (pre-segmented,
pre-classified areial photos and thematic maps (see
\ref{subsec:reference_data_and_semantic_characterisation}) and the attachment of
subsequent semantic characteristics (see 
\ref{subsec:reference_data_and_semantic_characterisation}), (2)
the selection of training and validation data (see
\ref{subsec:Selection_of_training_validation_data} the generation of
classification rules using machine learning algorithms and finally (4) the
ontology-based classification process (see \ref{subsec:Onto_classification}). 

The software relies on a PostgreSQL database and various open source Python and 
Java libraries to interact with the database, convert files and execute a 
reasoner over the created \gls{owl} file (see 
\ref{subsec:Onto_classification}). The \gls{owl}API 
\footnote{\url{https://github.com/owlcs/owlapi}} is used to interact with the 
\gls{owl} files and execute the FaCT++ reasoner ~\citep{Tsarkov2006}. Currently 
the rule generation (see \ref{subsec:rulegen_data_mining}) module uses two 
algorithms: \gls{dt}\footnote{\url{ 
http://scikit-learn.org/stable/modules/tree.html\#tree} } and one additional 
algorithm called the Separability and Thresholds (SEaTH) algorithm 
~\citep{Nussbaum2006}. The \gls{et}\footnote{\url{ 
http://scikit-learn.org/stable/modules/generated/sklearn.ensemble.ExtraTreesClas
 sifier.html}} is used as a reference classification.
\begin{figure}
\includegraphics[width=1\linewidth]{diagrams/final_workflow_diagram.png}
\caption
    {
        1) The thematic maps (biotope, soil, etc.) are unioned together and
        statistics are calculated for each combined polygon.
        2) Training and testing data are created and saved in the database.
        3) The rules are generated for e.g.\ the ``wetness'' indicator from
        training data.
        4) The rules are imported into the \gls{owl} ontology along with the 
testing
        data as individuals. The reasoner performs A-box reasoning to determine
        class membership.
    \label{fig:full_workflow}}
\end{figure}
\label{subsec:reference_data_and_semantic_characterisation} 
\subsection{Reference data preparation and semantic characterization}
The iterative object-based image analysis is performed by ~\cite{Tintrup2015} 
using Defiens eCognition Server in a multi-resolution approach 
~\citep{baatz2001ecognition} taking into account thresholds. The data is 
segmented by multi-spectral (B, G, R, NIR) orthophotos (see 
\ref{sec:usecase_data}) spectral information and indices (Bare Area Index, 
etc.) and height from the DEM to separate between biotic and abiotic features 
~\citep{Tintrup2015}. The detailed data preparation workflow is shown in figure 
\ref{fig:pre-processing}. This segmentation follows the \gls{eagle} concept and
captures land cover classes and labels created polygons as such. This step is
crucial as the quality of the segmentation determines the quality of the later
identification step as the value of an indicator such as wetness for a grassland
might differ for that of a forest. Using \gls{eunis} class descriptions and
interpretation guidelines ~\citep{EUNISManual}, a set of indicators 
tailored to
the available input data for our test area were created. Detailed analysis of
the \gls{eunis} nomenclature showed that some indicators do not lend themselves 
to
easily be detected by \gls{rs} data. Therefore consulting with an expert 
ecologist,
indicators were added to produce a meaningful formalization of the classes. The
formalization can be written to an \gls{owl} ontology, which is able to store 
complex
logical connections (axioms) in an \gls{owl} file.

An \gls{owl} ontology is composed of classes, individuals and 
properties. Classes are
sets of individuals and properties come in two forms: an object property defines
a relationship between two individuals and a data property  places a data type
constraint on the individual ~\citep{OWL2}. Furthermore, a reasoner can 
perform
A-Box and T-Box reasoning over the defined axioms. 

Relationships between objects (subsumption, disjointedness, etc). are 
discovered and checked for consistency during T-box reasoning. Using software, 
such as Protege, users can see how the defined axioms are related in the stored 
knowledge base (\gls{owl}) and check for logical consistency. If OWLIndividuals 
(objects) are in the knowledge base then A-Box reasoning can also be performed 
finding if any of the individuals fit into the defined classes. As \gls{eunis} 
is an hierarchically structured classification scheme testing that the 
formalization follows the same hierarchy is important. 
%LC

We use environmental variables (e.g., wetness, soil maturity, vegetation type,
etc.) from the classification schemes and concepts from \gls{eagle} to preserve
interoperability by using this well-formalized vocabulary. All used indicators
for this research are shown in Table \ref{tab:indicators_classes}. An example of
a \gls{eunis} class, E2.22 ``Sub-Atlantic lowland hay meadows'' modelled with 
selected
remote sensing indicators is written in description logic (DL) below (see
equation ~\ref{eq:description_logic}).
\begin{equation}\label{eq:description_logic}
\begin{align*}
%\begin{split}
    E2.22 &\equiv \exists has\_wetness \{``mesic''\} \\
    &\qquad {} \cap \exists has\_hydromorphic \{``false''\} \\
    &\qquad {} \cap \exists has\_immature\_soil \{``false''\} \\
    &\qquad {} \cap \exists species\_richness \{``false''\} \\
    &\qquad {} \cap \exists has\_usage \{``mowing''\} \\
    &\qquad {} \cap \exists has\_usage\_intensity \{``medium''\} \\
\end{align*}
\end{equation}
\label{subsec:Selection_of_training_validation_data}

%\label{LU_anthropomorphic}
Basic land cover extraction is performed
by the segmentation process. 
These polygons include spectral information and various statistics and indices
such as SAGA wetness index, wind effect, etc. This follows the \gls{eagle} data
model's separation of land use and land cover information.

To generate the reference data, we combined agricultural, foresty, biotope and 
geological data from different sources, creating a comprehensive database for 
\gls{rlp}. These thematic maps include anthropomorphic characteristics of the 
\gls{lu} such as cultivation and management practice (grazing or mowing) as 
well as biophysical characteristics (wetness, soil composition, morphology 
etc.). To ensure the comparability of these base data sets they were 
characterized with a common reference model of indicators, which have been 
developed within the \gls{natflo} project. That means, all base data sets 
include the subsequent indicators (such as wetness, usage etc.) and can 
therefore be used as reference data. The included polygons are much larger than 
the polygons produced by the segmentation (see 
\ref{subsec:reference_data_and_semantic_characterisation}, so that all polygons 
completely within the thematically combined polygon receives all of its 
attributes. This step adds the class labels used for 
training the data mining algorithm.
\subsection{Selection of training and validation data}
The data was divided into 40\% for training and 60\% for validation. Training 
\gls{seath} starts with 15 objects per class being randomly selected from the 
training data. One benefit of \gls{seath} is that one does not need many 
training objects. In ~\cite{Nussbaum2006}, for example, the authors suggest 
using only very characteristic features for training and only used around 10 
samples per class. The authors also state that usually two features per class 
is enough to produce accurate results.

We use an evolutionary search algorithm with 10 evolutions based on the 
Distributed Evolutionary Algorithms in Python ~\citep{DEAP_JMLR2012} software 
to find the best features and used balanced class weights due to the very 
different distribution of grassland biotopes. Three different fitted \gls{dt} 
classifiers were compared: 1) a \gls{dt} fitted with the features chosen by the 
evolutionary search algorithm 2) a \gls{dt} trained on all features with 
parameters chosen after trial and error 3) a \gls{dt} trained on the top 10 
most important features as chosen by the \gls{et} classifier. Each \gls{dt} 
classifier undergoes a 5-fold cross validation to verify results and the 
\gls{dt} with the highest overall accuracy \gls{dt} was chosen.
\label{subsec:rulegen_data_mining}
\subsection{Rule generation}
Different data mining algorithms are trained on the training data to generate 
rules for the classification procedure. For the \gls{dt} case the tree is 
parsed according to the depth first search algorithm. Starting with the root 
node, the algorithm visits edges first, traveling as far as and only when it 
reaches a leaf node does it backtrack. The nodes contain the decision thresholds 
and all nodes visited before reaching the leaf node (the parents of that node) 
are chained with a logical ``AND'' together to create a rule. The leaf node also 
has information on how many objects were classified by going down this branch 
and how many were falsely classified. The rule is assigned to the class with 
the best accuracy at that leaf node. The process continues until all leaf nodes 
are visited. The rules are joined together by logical ``OR"s and become a rule 
set for the indicator under investigation. An example decision tree trained on 
immature soil is depicted in figure \ref{fig:decisiontree}.
%Die Graphik ist etwas klein. Vielleicht ist Querformat besser?!
\begin{figure}
\includegraphics[width=\textwidth]{diagrams/natfo_immature_soil_dt.png}
    \caption{An example decision tree trained on immature soil 
indicator.\label{fig:decisiontree}}
\end{figure}

\gls{seath} makes the assumption that the classes follow a normal distribution 
and determines the separability of the object classes by computing the 
pairwise Jeffries-Matusita distance. The distance outputs a score from 0-2 
signaling the quality of separation where 2 is total class separation with 
this feature. The \gls{seath} algorithm outputs rules that include the best 
features to separate one class from another. Since we have over 200 features and 
the authors of \gls{seath} suggest using only 2-3 features, we parse the first 
three lines as the list is sorted from best to worst. We join the features with 
their corresponding thresholds using logical ANDs. We do this for every class, 
chaining the preceding rule and the next rule with an OR. 

An example rule from the \gls{dt} is shown below. The subclass ``low'' from 
``usage intensity'' has a rule that is generated by the decision tree algorithm 
as seen in \ref{lst:dt_rule_snippet_csv} below. The first name is the 
feature/statistic's name, followed by a threshold. For the \gls{dt}, as in the 
example, the last number is the node in the tree where the threshold comes 
from. This information is currently not being used.
\lstset{language=XML,tabsize=2, label=lst:dt_rule_snippet_csv, caption=\lstname}
\begin{lstlisting}[frame=single]
    wief_max,>,1.092550,0
    tpi5_max,<=,-0.341000,420
    swi_median,<=,8.097600,421
    toin_max,<=,1536.561279,422
    pan4_glcm_std_135,<=,-7.890923,423
    b_std,<=,10.100981,424
\end{lstlisting}

\lstset{basicstyle=\footnotesize,language=XML,tabsize=2,
label=lst:dt_rule_snippet_owl , caption=\lstname, breaklines=true}
\begin{lstlisting}[frame=single,fontadjust]
            <DataSomeValuesFrom>
                <DataProperty IRI="#has_tpi5_max"/>
                <DatatypeRestriction>
                    <Datatype abbreviatedIRI="xsd:double"/>
                    <FacetRestriction facet="&xsd;maxInclusive">
                        <Literal datatypeIRI="&xsd;double">-0.341
                        </Literal>
                    </FacetRestriction>
                </DatatypeRestriction>
            </DataSomeValuesFrom>
            <DataSomeValuesFrom>
                <DataProperty IRI="#has_wief_max"/>
                <DatatypeRestriction>
                    <Datatype abbreviatedIRI="xsd:double"/>
                    <FacetRestriction facet="&xsd;minExclusive">
                        <Literal datatypeIRI="&xsd;double">1.09255
                        </Literal>
                    </FacetRestriction>
                </DatatypeRestriction>
            </DataSomeValuesFrom>
        </ObjectIntersectionOf>
    </ObjectUnionOf>
</EquivalentClasses>
\end{lstlisting}

\ref{lst:dt_rule_snippet_owl} shows how the first two lines of 
\ref{lst:dt_rule_snippet_csv} this rule look in an \gls{owl}/XML ontology. The 
rule has a collection of DataSomeValuesFrom within a nested datatype 
restriction 
corresponding to the rule threshold. A class can be defined by many rules 
containing multiple datatype properties that are chained together with logical 
AND (intersection) operators. The rules are then joined by a logical OR 
operator 
as can be seen in the outer ObjectUnionOf. 

\subsection{Ontology-based classification}
\label{subsec:Onto_classification}


After the algorithms produce rules for each indicator, these are then added as 
facet restrictions to an \gls{owl} ontology. The polygons from the testing 
table 
are 
loaded from the database as \gls{owl}Individuals into the same ontology and the 
reasoner FaCT++ classifies all polygons by applying A-box reasoning over the 
rules. The classification results by the reasoner is written to the database. 
\subsection{Validation} 
\label{subsec:Validation}
For \gls{seath} 15 objects per class were used from the training data 
because the algorithm performs better with a smaller number of features 
~\citep{Nussbaum2006}. We selected only polygons greater than 200m$^{2}$ 
identified to be herbaceous plants by the segmentation algorithm for training 
and testing. For training we selected only polygons where the value for the 90 
percentile of the \gls{ndsm} less than 1 meter. Therefore the testing data 
includes polygons of grasslands between trees in orchards and grasslands in 
agricultural production. Finally, to evaluate the quality of the results in 
respect to well-established, popular classification approaches the outcomes 
were compared to an \gls{et} reference classification (see 
\ref{tab:accuracy_indicators}
We use precision\footnote{\url{
http://scikit-learn.org/stable/modules/generated/sklearn.metrics.precision_score
.html\#sklearn.metrics.precision\_score}} (positive predicative value), 
recall\footnote{\url{
http://scikit-learn.org/stable/modules/generated/sklearn.metrics.recall_score.ht
ml\#sklearn.metrics.recall\_score}} (sensitivity, true positive rate), 
value) and 
f-score\footnote{\url{
http://scikit-learn.org/stable/modules/generated/sklearn.metrics.f1_score.html\#
s
klearn.metrics.f1\_score}} to determine the accuracy of classification results. 
The precision score (\ref{eq:precision}) is a reflection of how many of the 
objects that were classified were true positives. Recall \ref{eq:recall} shows 
the classifier's ability to find all relevant objects (positive samples). The 
f-score is the harmonic mean of the precision and recall.
\begin{equation}\label{eq:precision}
\begin{align*}
    \frac{\text{true positives}}{\text{true positives + false positives}}
\end{align*}
\end{equation}

\begin{equation}\label{eq:recall}
\begin{align*}
    \frac{\text{true positives}}{\text{true positives + false negatives}}
\end{align*}
\end{equation}
\begin{equation}\label{eq:fscore}
\begin{align*}
    2 * \frac{\text{precision * recall}}{\text{precision + recall}}
\end{align*}
\end{equation}
First the rules are generated for each indicator and the metrics in
\ref{eq:precision}, \ref{eq:recall} and \ref{eq:fscore} are saved for accuracy
analysis and for later scrutiny. The results for each indicator is saved in the
database. The indicators are then aggregated using SQL queries according to the
formalized \gls{eunis} class definition to find which objects meet the 
requirement.
Objects that meet the rule as defined in \ref{tab:indicators_classes} are
assigned to that class. The classified result is then compared to the actual
class label to determine the quality of the results.

\section{Results}
\begin{table}
    \centering
    %\rowcolors{2}{lightgray}{white}
    \begin{tabular}{l l c c c c}
    Indicator & Algorithm & Precision & Recall & F-score & 
    Support\\
    \hline
    \multirow{3}{*}{hydromorphic}
    & DT & 0.86 & 0.92 & 0.89 & 21098\\
    & SEaTH & 0.86 & 0.87 & 0.86 & 20067\\
    & ET & 1.0 & 1.0 & 1.0 & 21098\\
    \cline{2-6}
    \multirow{3}{*}{immature soil}
    & DT & 0.99 & 0.99 & 0.99 & 21098\\
    & SEaTH & 0.99 & 0.42 & 0.58 & 17883\\
    & ET & 1.0 & 1.0 & 0.99 & 21098\\
    \cline{2-6}
    \multirow{3}{*}{species richness}
    & DT & 0.08 & 0.25 & 0.12 & 21098\\
    & SEaTH & 0.11 & 0.31 & 0.16 & 16789\\
    & ET & 0.08 & 0.28 & 0.12 & 21098\\
    \cline{2-6}
    \multirow{3}{*}{usage}
    & DT & 0.34 & 0.45 & 0.38 & 21098\\
    & SEaTH & 0.27 & 0.16 & 0.12 & 18876\\
    & ET & 0.36 & 0.54 & 0.40 & 21098\\
    \cline{2-6}
    \multirow{3}{*}{usage intensity}
    & DT & 0.22 & 0.2 & 0.17 & 21098\\
    & SEaTH & 0.26 & 0.38 & 0.28 & 14245\\
    & ET & 0.34 & 0.46 & 0.36 & 21098\\
    \cline{2-6}
    \multirow{3}{*}{wetness}
    & DT & 0.4 & 0.62 & 0.49 & 21098\\
    & SEaTH & 0.45 & 0.35 & 0.39 & 12178\\
    & ET & 0.41 & 0.63 & 0.49 & 21098\\
    \end{tabular}
    \caption{Overall classification accuracy of 
indicators\label{tab:accuracy_indicators}}
\end{table}
The poor performance of the wetness indicator negatively effects classification 
of the
biotopes. Luckily, the immature soil and hydromorphic indicators when combined
are disjoint (see \ref{tab:indicators_classes})
\subsection{EUNIS Level 2 Classification}
\label{subsec:level2_classification}
The \gls{dt} classifier over estimates the presence of E2 grasslands whereas 
the SEaTH
algorithm underestimates their presence. The \gls{dt} classifier has 4115 false
positives (19654 classified vs. 15539 actual) whereas SEaTH classified only 1511
objects as E2 (14028). % Der Absatz ist gut!
\begin{table}
\centering
\begin{tabular}{c c c c c}
Class & Precision & Recall & F-score & Support\\
\hline
E1 & 0.5 & 0.01 & 0.02 & 122\\
E2 & 0.79 & 0.58 & 0.67 & 15539\\
E3 & 0.7 & 0.11 & 0.2 & 61\\
unclass & 0.0 & 0.0 & 0.0 & 5376\\
avg & 0.59 & 0.43 & 0.49 & 21098\\
\end{tabular}
\caption{Classification accuracy of grasslands using the \gls{dt} 
classifier\label{fig:dt_lvl2_classification}}
\end{table}
The largest number of polygons in the testing data not belonging to \gls{eunis}
grasslands belong to I1 which are described as ``Arable land and
market gardens''. The class includes meadow/pastureland under management and
recently fallowed land under I1.5. The high recall and low precision score
especially for wetness 

The dominance of mesic grasslands in Saarburg is clearly reflected in
the results. In the testing data of the 21098 objects, 15539 objects - 74\% -
are mesic grasslands. Less than 1\% of objects are in E1 (122) and E3 (61). 
\begin{table}
\centering
\begin{tabular}{c c c c c}
Class & Precision & Recall & F-score & Support\\
\hline
E1 & 0.03 & 0.68 & 0.05 & 122\\
E2 & 0.68 & 0.01 & 0.02 & 15539\\
E3 & 0.02 & 0.38 & 0.04 & 61\\
unclass & 0.0 & 0.0 & 0.0 & 5376\\
avg & 0.5 & 0.01 & 0.01 & 21098\\
\end{tabular}
\caption{Classification accuracy of grasslands using the \gls{seath}
algorithm\label{fig:seath_lvl2_classification}}
\end{table}

\subsection{EUNIS Level 3 classificaton}
\begin{table}
\centering
\begin{tabular}{c c c c c}
Class & Precision & Recall & F-score & Support\\
\hline
E1.2 & 0.5 & 0.01 & 0.02 & 122\\
E2.1 & 0.63 & 0.69 & 0.66 & 11509\\
E2.6 & 0.17 & 0.04 & 0.07 & 4335\\
E3.4 & 1.0 & 0.03 & 0.06 & 64\\
avg & 0.37 & 0.37 & 0.36 & 21788\\
\end{tabular}
\caption{Classification accuracy for \gls{eunis} level 3 using \gls{dt}}
\end{table}
\begin{table}
\centering
\begin{tabular}{c c c c c}
Class & Precision & Recall & F-score & Support\\
\hline
E1.2 & 1.0 & 0.01 & 0.02 & 122\\
E2.1 & 0.0 & 0.0 & 0.0 & 11105\\
E2.6 & 0.5 & 0.0 & 0.0 & 4240\\
E3.4 & 0.02 & 0.3 & 0.04 & 61\\
unclass & 0.0 & 0.0 & 0.0 & 5374\\
avg & 0.11 & 0.0 & 0.0 & 21098\\
\end{tabular}
\caption{Classification accuracy for \gls{eunis} level 3 using SEaTH}
\end{table}
At level three fewer objects are found and classified. Our indicators are not 
capable of detecting the differences at this level. For instance in dry 
grasslands there is only one object in E1.71, all the rest of the dry 
grasslands are in E1.262.

\section{Discussion}
As can been seen in table \ref{tab:accuracy_indicators}, the immature soil and
hydromorphic indicators performed well. The poor performance of the wetness
indicator in particular is surprising as the data mining algorithms had 
access to many different wetness indicators and morphological data. Since 
classifying \gls{eunis} level 2 grasslands requires information on wetness, 
the classification suffers. The performance of usage and usage intensity is 
most likely due to a lack of time series data. In regards to algorithmic 
performance, SEaTH performed poorly overall and the results show that the 
\gls{dt} classifier achieves comparable results to the \gls{et} classifier 
for the indicators chosen. 

The very few dry grasslands in the Saarburg region hurt the classifier's 
ability to learn from the data. Only one dry grassland classified as such fits 
the rule's definition of a dry grassland. The reasons for this could be due the 
differences in data quality between the biotope map and the agriculture map. 
It could also be a conceptualization problem with the translation from the 
\gls{rlp} classification to \gls{eunis}.

Visually inspecting the data revealed that polygons were in fact grasslands, 
agricultural land or meadows. Most belonged the class I1 - ``Arable land and 
market gardens'' and the \gls{eunis} biotope classification guide specifically 
states that turf and sports fields are excluded. Since I1 is ``species poor'' 
adding ``species richness'' to the E2 rule reduced the number of polygons 
classified as I1 but also reduced the total number of E2 classes as well. Thus, 
adding species richness was not helpful. The height restriction and the class 
name ``herbaceous plants'' from the segmentation properly excluded trees, 
artificial buildings, etc. The larger size of the thematic map's polygons means 
that a polygon of grasslands in between an orchard will still be assigned the 
orchard's label and \gls{eunis} indicators. Early on the issue with orchards 
was identified as a troubling area for segmentation and this has not yet been 
resolved. A multi-scale approach will probably be needed to solve this problem. 

The selection of grasslands greater than 200m$^{2}$ in size might have also led 
to errors as max and min values in the \gls{dsm} had a large range. The 
pre-segmentation class herbaceous plants could be too lenient as shrubs were 
found in polygons labeled as herbaceous. In addition other indicators are 
needed to more accurately differentiate between grasslands.

Lastly, properly exploiting a dimension reduction strategy using principal 
component analysis or some other method  could possibly improve the data mining 
algorithms classification accuracy. The addition of time series data for 
intensity and soil maps would greatly help classification accuracy.  
\section{Conclusion}
The results demonstrate that the method provides additional value when compared 
with a normal classification approach and that with proper conceptualization, 
reference data and dimension reduction could achieve results comparable to an 
\gls{et} classifier. Furthermore, we tested the method on formalized 
\gls{eunis} grasslands and showed that it could be used to refine the 
indicators and revisit the conceptualization. 

Further investigation of the segmented polygons and a test area with a more
uniform distribution of biotope classes is needed. Accurate reference data is
needed to diagnose why the wetness indicator performed so poorly. Other 
algorithms should be applied for comparison.  

We demonstrated that an ontological-based semi-automated classification system
produces interoperable and exchangeable classification outputs
and results and is not constrained by data mining
algorithms or software. Furthermore, such tools can be applied to different 
domains and, for biodiversity monitoring, help decision-makers make informed 
decision by being able to compare outputs.
\section{Acknowledgments}
We would like to thank \gls{rlp} AgroScience GmbH for processing the data and 
the \gls{rlp}'s Environment Ministry for funding the \gls{natflo} project. This 
work was conducted using the Prot\'eg\'e resource, which is supported by grant 
GM10331601 from the National Institute of General Medical Sciences of the 
United States National Institutes of Health.
\section{Appendix}
The \gls{owl} ontology with rules generated by the \gls{dt} algorithm used in 
this study is
located at: \url{http://www.user.tu-berlin.de/niklasmoran/grassland_dt.owl} and
the SEaTH \gls{owl} file is located:
\url{http://www.user.tu-berlin.de/niklasmoran/grassland_seath.owl}
\bibliographystyle{model2-names}
\section{References}
\bibliography{references}
%\printglossary
\end{document}
