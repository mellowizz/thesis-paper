\documentclass[authoryear, review,12pt,number]{elsarticle}
%\documentclass[authoryear, preprint,12pt,number]{elsarticle}
\usepackage[numbers]{natbib}
\usepackage{graphicx}
\usepackage{float}
\usepackage{rotating}
\usepackage{stfloats}
\usepackage{lineno}
\usepackage[linesnumbered,ruled,vlined]{algorithm2e}
\usepackage{tabulary}
\usepackage{graphicx}
\usepackage{color}
\usepackage[none]{hyphenat} \usepackage[table]{xcolor} \sloppy
\usepackage{hyperref}
\usepackage{amsmath}
\usepackage{multirow}
\usepackage{rotating}
\usepackage{adjustbox}
\usepackage{graphicx}% http://ctan.org/pkg/graphicx
\usepackage{booktabs}% http://ctan.org/pkg/booktabs
\usepackage{xparse}% http://ctan.org/pkg/xparse
\usepackage{booktabs}
\usepackage{array}
\usepackage[ddmmyyyy]{datetime}
\renewcommand{\dateseparator}{.}

\newcommand*\OK{\ding{51}}
\newcolumntype{R}[2]{%
    >{\adjustbox{angle=#1,lap=\width-(#2)}\bgroup}%
    l%
    <{\egroup}%
}
\newcommand*\rot{\multicolumn{1}{R{60}{1em}}}% no optional argument here,
% please!
\begin{document}
\begin{titlepage}
\begin{center}

% Upper part of the page. The '~' is needed because \\
% only works if a paragraph has started.
\textsc{\LARGE Technische Universit\"at Berlin}\\[0.5cm]
\textsc{Master Thesis}\\[1.5cm]

\textsc{\Large Classifying EUNIS Habitats using Ontologies and Data Mining
Methods}\\[1.5cm]
\textsc{\textit{A thesis submitted in partial fulfillment of the requirements
for the degree of}}\\[1.25cm]
\textsc{\Large Master of Science in Environmental Planning}\\[1.5cm]
\textsc{Faculty VI - Planning Building Environment\\
 Geoinformation in Environmental Planning}\\[1.5cm]

% Title
% Author and supervisor
\noindent
\begin{minipage}{0.5\textwidth}
\begin{flushleft} \large
\emph{Author:}\\
T. Niklas \textsc{Moran}
\end{flushleft}
\end{minipage}%
\begin{minipage}{0.5\textwidth}
\begin{flushright} \large
\emph{Supervisor:} \\
Prof. Dr.~Birgit \textsc{Kleinschmit}
Mag. rer. nat.~Simon \textsc{Nieland}
\end{flushright}
\end{minipage}

\vfill

% Bottom of the page
{\large January 2015}

\end{center}
\end{titlepage}

\begin{titlepage}
\begin{center}
{Author's Declaration}
\end{center}
I hereby certify that I am the sole author of this master thesis. 
Furthermore, I confirm that no sources have been used in the preparation of this thesis other
than those indicated in the thesis itself. The works of other people included in
my thesis, published or otherwise, are fully acknowledged in accordance with the
standard referencing practices. This thesis has not been submitted for another
degree or master to any other University or Institution.
\\
\\
Hiermit versichere
ich, dass ich die vorliegende Arbeit selbstst�ndig verfasst und keine anderen als die
angegebenen Quellen und Hilfsmittel benutzt habe. Alle Ausf�hrungen, die anderen
ver�ffentlichten oder nicht ver�ffentlichten Schriften w�rtlich oder
sinngem\"a\ss entnommen wurden, habe ich kenntlich gemacht. Die Arbeit hat in
gleicher oder \"ahnlicher Fassung noch keiner anderen Pr\"ufungsbeh\"orde vorgelegen.

\vfill
\begin{center}
\noindent
\begin{minipage}{0.5\textwidth}
\begin{flushleft}
\rule{5cm}{0.4pt}
Date/Datum 
\end{flushleft}
\end{minipage}%
\begin{minipage}{0.5\textwidth}
\begin{flushright} 
\rule{5cm}{0.4pt}
Signature/Unterschrift
\end{flushright}
\end{minipage}


\end{center}
\end{titlepage}

\begin{frontmatter}
\linenumbers
\title{Classifying EUNIS habitats using ontologies and data mining methods}


\author[TUB]{Niklas Moran\corref{cor1}}
\ead{niklasmoran@mailbox.tu-berlin.de}

\author[TUB]{Simon Nieland}
\author[TUB]{Birgit Kleinschmit}

%\author[TUB]{Michael F\"orster}

\address[TUB]{Geoinformation in Environmental Planning Lab, Technische
Universit\"at Berlin, Stra\ss e des 17. Juni 145, 10623 Berlin, Germany}

\cortext[cor1]{Corresponding author at: Geoinformation in Environmental Planning
Lab, Technische Universit\"at Berlin, Stra\ss e des 17. Juni 145, 10623 Berlin,
Germany}  % el.:+49 30 314 72601;}

\begin{abstract}
Remote sensing Biodiversity monitoring and 
\end{abstract}

\begin{keyword}
remote sensing, biotope classification, data mining,
generalisation, nature conservation, OWL, EUNIS, OBIA
\end{keyword}

\end{frontmatter}

\linenumbers

\section{Introduction}
Recognizing the importance of functioning habitats to reduce biodiversity loss, 
the European Union has implemented an environmental conservation framework to 
protect and conserve vital habitats in accordance with the Convention on 
Biological Diversity. An integral part of this framework is the EU 
Habitats Directive (Council Directive) 92/43/EEC [1992], which established the 
Natura 2000 network of habitats. The law requires conservation and monitoring 
of designated habitats by member states and for a report to be submitted every 
six years. Environmental data to determine biodiversity status must be 
collected to comply with the statute. Yet, comparing data used for these reports 
can be difficult due to varying data collection methods by the nature 
conservation authorities in each member state \citep{INSPIREdataspecs, INSPIRE}. 
The main issue lies in the subjective nature of field surveys to identify 
habitats \citep{Cherrill1999, Cherrill1999a, Nieland2015}. Furthermore, habitat 
status is mostly generated in bottom-up approaches taking into account the 
national and regional interpretation guidelines \citep{INSPIREdataspecs}. This 
subjective and time-consuming task of conducting field surveys could be 
partially replaced with an automated remote sensing (RS) method that uses  
Geographic Object-Based Image Analaysis(GEOBIA) to reduce the subjectiveness,
reduce costs and increase temporal resolution.
\\
Remote sensing combined with GEOBIA, which segments the image into homogeneous 
image objects \citep{Blaschke2010}, offers opportunities to automate and 
collect large amounts of computer-readable data useful for nature conservation 
and biodiversity monitoring \citep{Corbane2015, VandenBorre2011, Mayer2011}. 
Yet, RS image analysis implicitly incorporates the expertise and knowledge of 
the person performing the analysis and reduces objectivity. This can be divided 
into remote sensing knowledge (spectral signature, remote sensing indices, etc.) 
and field knowledge (feature properties, spatial relations, etc) 
\citep{Andres2013a}. This knowledge is often neither completely nor explicitly 
defined as it is based on trial and error but influences the classification. To 
ensure accuracy and applicability of classification outputs for conservation, 
experts with detailed knowledge of the sites are needed to interpret the EO 
data. The distance between the high-level semantics used by the expert to 
describe events and domain concepts and the low-level information quantified 
from data is reffered to as the ``semantic gap''.
%vielleicht findest du hierf�r noch eine Quellee
An ontology, defined as ``an explicit specification of a conceptualization''
\citep{gruber1993} could help bridge the ``semantic gap'' and allow for
better data transferability, predictability, interchangeability, knowledge
and workflow management and logical consistency. 
% citations!
% tranerability, predictablity usw?. Vielleicht kannst du das auch noch etwas
% genauer ausf�hren. Ich w�rde auch noch OWL einf�hren als Speicherstruktur des
% semantic web und M�glichkeit zur formalisierung.
Moreover, through the use of reasoners (inference engines) 
that infer logical consequences over axioms and asserted facts one can discover 
new knowledge \citep{Arvor2013, Andres2013a}. RS and field expert knowledge can 
be digitized in ontologies, thus allowing for a hierarchy of concepts for 
improved automatic image annotation and retrieval using concepts from both 
fields to produce more accurate results 
\cite{Srikanth:2005:EOA:1076034.1076128}. Janowicz \cite{Janowicz2012} advocates 
for more observation-driven ontologies and for including machine learning, 
statistics and data mining to construct ontological primitives. While research 
on using observation-based ontologies for classifications is limited, the 
available research in RS using ontologies is briefly summarized below. \\
Ontologies modeled on the Land Cover Classification
System and the General Habitat Category were integrated into tools used to 
monitor and protect areas in the EU \citep{Arvor2013}. The authors note 
that using the taxonomy of the different classification systems
makes it possible to include expert knowledge in the process. 
Further research included expert knowledge from remote sensing and ecology to 
classify Natura 2000 sites using a hierarchical classification system. The 
research used pixel-based analysis and OBIA for greater classification accuracy 
but still relies on a rule-base created by an expert\citep{Lucas2015}. Other 
research includes classification of urban building types using a three-layered 
architecture \citep{diSciascio2013} and a semi-automated classification of 
urban building using the Random Forest (RF) classifier to determine variable 
importance of features from airborne laser scanner data \citep{Belgiu2014}. 
Ontologies have also been paired with different algorithms to automatically 
acquire classification rules: a genetic programming algorithm 
\citep{Forestier2012470} and the C4.5 data mining algorithm 
\citep{Sheeren2006ML}.
Harnessing ontologies to improve spatial data interoperability has been 
demonstrated by interoperability has been demonstrated by \cite{Nieland2015}. 
%% Hier sollte auf jeden Fall noch die Arbeit von den Griechen rein die ich die
% geschickt habe.
\\
Even though researchers recently developed a number of indicators using
different sensors for habitat evaluation \citep{Nagendra2013}, classification
procedures and rule-sets were not formalized to be computer readable and
therefore suffer from similar transferablity and reproducablity problems as
manual habitat mapping \citep{Arvor2013, Nieland2015}.
Therefore a formalized computer-readable ontology could help solve these
problems and allow scientists to see how the classification was performed and 
be aware of possible incompatibilities before combining datasets
\citep{Janowicz2012}. 
Furthermore, there is no standardized set of indicators
using RS for trans-national habitat evaluation \citep{Lucas2015}. Therefore,
technical solutions to increase interoperability by thematically harmonizing
environmental data and systematize data collection methods from remote sensing
inputs in an automated workflow is needed. This would allow decision-makers 
to better assess and compare outcomes.
\\
In this paper we propose an automated system that can
classify selected biotopes according to the European Union Nature Information
System (EUNIS) biotope classification schema using EO data, existing
thematic maps (biotope, forestry, etc.) and expert knowledge formalized in an
ontology. 
%Es muss auch noch ein kleiner Absatz zum EAGLE Datenmodell rein. Was machen
% die? Warum nehmen wir das?
The combination of data mining algorithms with ontology-based classification
has not yet been done and is a first in remote sensing research. This method
contributes to the goal of empirically-derived rule creation and enhances
data interoperability and comparison as proposed by Janowicz
\citep{Janowicz2012}.
%The biotope indicators serve as class labels for training data mining
%algorithms which generate rules that can be imported into an ontology. Once the
%rules and segmented objects are imported into the ontology, a reasoner can
%perform reasoning classifying which objects belong to which class (A-Box 
%reasoning) and how the classes are related (T-box reasoning).
%Das sollte eher in die Methode
%Versuche doch die Kernthesen in max. 2-3 S�tzen zusammen zu fassen (vielleicht
% mit bullet points).
% Darauf kannst du dann in der Conclusion wieder zur�ckkommen. 


\section{Method}
This section describes the developed method for classifying EUNIS habitats by
using data mining algorithms to generate rules based on remote sensing, digital
surface model (DSM) and digital terrain model (DTM) derived statistics of
pre-segmented polygons. The method's use of ontologies and the advantage
therein is also explained.
\subsection{Overview of the Automated EUNIS Habitat Mapping System}
Figure ~\ref{fig_full_workflow} gives an overview of the method. We combined
agricultural, foresty, biotope and geological data from different sources
and after controlling for agreement were merged into a comprehensive database
for all of Rhineland Palatinate. 
The developed system is comprised of (1) attaching
formalized habitat indicators from expert knowledge to pre-segmented polygons,
(2) rule generation by running data mining algorithms over randomly selected
polygons and using the habitat indicators as class labels, (3) importing rules,
EUNIS classes and polygon attributes into an ontology and finally (4)
classifying polygons with the Fact++ reasoner \citep{Tsarkov2006} and writing
results back to the database. 
An overview of the complete workflow is shown in
figure ~\ref{fig_full_workflow}.\\ %Diesen Teil musst du noch genauer
% beschreiben. Das ist der Kern deiner Arbeit. Auch wenn du dich in den n�chsten
% Unterkapiteln nochmal wiederholst. Es ist sehr wichtig f�r das Verst�ndnis,
% dass du hier alles detalliert beschreibst. Ich w�rde f�r jeden Punkt
% mindestens einen Satz schreiben.
% Verweise auf die anderen Kapitel (z.b. (1) ist beschrieben in 2.2 und 2.3 usw.).
% Hier muss noch rein: Woher kommen die
% Ausgangsdaten? Wie wurden Referenzdaten erstellt?(Verweise auf Kapitel 2.2)
% Wie funktioniert das Sampling? Wie formalisierst du die Regeln (OWL Axioms)?
% Schreibe, dass wetness in der Graphik nur ein Beispiel ist und du das f�r
% verschieden Indikatoren anwenden willst. Erkl�re A-Box und T-Box reasoning.
%Erkl�re Expert knowledge. Wer macht da was? 
The software relies on a PostgreSQL database back-end with PostGIS extensions
enabled and various open source Python and Java libraries to interact with the
database, convert files and execute a reasoner over the created OWL file.
%SQLAlchemy is used for database interaction and data management is done with
%the Python Data Analysis Library (Pandas) \citep{McKinney2010}. 
The OWLAPI is used to interact with the OWL files and execute the FaCT++
reasoner \citep{Tsarkov2006}. The software is freely available and is released
under an open source license. 
%Wie gesagt: Welche Bibliotheken du benutzt hast ist nicht so wichtig.
>>>>>>> f3420cde4bd708bf0ecf446f694a672db87437a6
\begin{figure}
\includegraphics[width=1\linewidth]{diagrams/another_workflow_diagram_large.png}
\caption
    {
        1) The thematic maps (biotope, forestry, etc.) are joined together and
        statistics are calculated for each combined polygon.
        2) Training and testing data are created and saved in the database.
        3) The training data is loaded into the data mining module and the
        rules are generated for e.g.\ the ``wetness" indicator
        4) The rules are imported into the OWL ontology along with the testing
        data as individuals. The reasoner performs A-box reasoning to determine
        class membership.
    } 
\label{fig_full_workflow}
\end{figure}
\subsection{Target indicators and base nomenclatures}
We tested our method on dry, mesic and wet grasslands (EUNIS classes E1, E2 and
E3 respectively). We adapted the selected EUNIS indicators to meet the
requirements of remote sensing analysis. We also adopted EAGLE nomenclature when
possible to increase interoperability and further re-use. The table
~\ref{tab_indicators_classes} describes the indicators that are needed to
identify and hence classify the biotopes according to the EUNIS classification
hierarchy. The EUNIS classification hierarchy is built upon different
environmental indicators to differentiate between classes. 

\begin{table}
\centering
  \begin{tabular}{clcccccccc}
  \rot{description}&\rot{EUNIS class}&\rot{wetness} & \rot{vegetation type} &
  \rot{usage} & \rot{usage intensity} & \rot{immature
  soil} & \rot{hydromorphic} & \rot{species richness} \\ \hline
\multirow{3}{*}{{Dry grasslands}}
    & E1   & dry & g/h & g/m/- & low & 1 & 0 & 0/1/- \\ 
    & E1.2 & dry & g/h & g/m/- & low & 1 & 0 & 1\\
    & E1.7 & dry & g/h & - & low & 1 & 0 & -/0\\ 
\multirow{4}{*}{Mesic grasslands} 
    & E2   & mesic & g/h & g/m & low/medium & 0 & 0 & 0/1/-\\
    & E2.1 & mesic & g/h & g & medium/high & 0 & 0 & -/0/1 \\
    & E2.6 & mesic & g & g/m/- & high & 0 & 0 & 0 \\
    & E2.7 & mesic & g/h & - & - & 0 & 0 & 0/1/- \\
\multirow{3}{*}{Wet grasslands}
    & E3   & very wet & g/h & g/m/- & medium & 0 & 1 & 1 \\
    & E3.4 & very wet & g/h & g/m & medium & 0 & 1 & 0/1 \\
    & E3.41 & very wet & g/h & m & medium & 0 & 1 & 1 \\
\end{tabular}
\caption{A `/ ' denotes OR and a '-' denotes 'none' and the first letter of 
each value is used to safe space. Vegetation type: {graminaceous, 
herbaceous}, usage: {grazing, mowing}. 
}
\label{tab_indicators_classes}
\end{table}
\label{subsec_indicators_and_nomenclatures}
\subsection{Formalization of Expert Knowledge}
EUNIS class descriptions and interpretation guidelines \citep{EUNISManual} were
used to develop a set of indicators to accurately formalize certain habitats
with regard to the infer-ability with available input data sets (see
~\ref{subsec_indicators_and_nomenclatures}). This formalization can then be
written to an OWL ontology, which is able to store complex logical connections
in a XML-based OWL2 file. The goal is to produce what Janowicz describes
as a ``micro theory \citep{Janowicz2012} which can then be used to map other
theories by training the data mining algorithms for the new
region and using the reasoner. We hope that once good test sites are taken for 
a region (for instance the federal state of Rhineland Palatinate) that the data 
mining algorithms will, at least for certain indicators, not need to be
re-trained, speeding up classification and makes generation of new reference
data unnecessary. This refers especially to the ones, which are derived from
rather stable data sources like the indices of the DEM/DSM\@. We use environmental
variables (e.g., wetness, dominant species%?soil maturity?
, etc.) from the classification schemes and concepts from EAGLE %Erst erkl�ren
to preserve interoperability by using this well-formalized
vocabulary.
% detailed examintation showed that indicators that are taken directly from the EUNIS
% nomenclature are not always suiteable for a RS-based analysis.
% Therefore certain indicators were adopted and added to be able to generate a
% meaningful formalization. All used indicators are shown in Table
% \ref{tab_indicators_classes}
 An example of a EUNIS class, E2.22
``Sub-Atlantic lowland hay meadows'' modelled with selected remote sensing
>>>>>>> f3420cde4bd708bf0ecf446f694a672db87437a6
indicators is written in description logic (DL) below
~\ref{eq:description_logic}.
\begin{equation}
\begin{align*}
%\begin{split}
E2.22 &\equiv wetness \exists \{``mesic''\} \\
&\qquad {} \land hydromorphic \exists \{``false''\} \\
&\qquad {} \land immature\_soil \exists \{``false''\} \\
&\qquad {} \land species\_richness \exists \{``false''\} \\
&\qquad {} \land usage \exists \{``mowing''\} \\
&\qquad {} \land usage\_intensity \exists \{``medium''\} \\
\end{align*}
\label{eq:description_logic}
\end{equation}

\subsection{Rule Generation with Data Mining ALgorithms}
First the data mining module splits data into training and testing data. 
The data mining module uses either the decision tree classifier 
(DT)\footnote{http://scikit-learn.org/stable/modules/tree.html#tree} 
\citep{scikit-learn} or the Separability and Thresholds (SEaTH) algorithm
\citep{Nussbaum2006} but the method can be extended to use other algorithms as
long as the algorithm can generate a rule-set that can be converted to OWL
queries. 
%Beschreibe genau wie du die Regeln aus dem DT raus bekommst (du gehst durch
% den Baum und schreibst die Werte raus, verbindest sie dann mit einem
% logischen UND usw.).Das gleiche f�r SeATH. Das ist der Knackpunkt deiner
% Arbeit.
After the algorithms produce rules for 
each indicator, these are then added as facet restrictions to an OWL ontology. 
The polygons from the testing table are loaded from the database as
OWLIndividuals into the same ontology and the reasoner FaCT++ classifies all 
polygons according to the rules. The classification results by the reasoner 
is written to the database. The full workflow is shown in figure 
\ref{fig_full_workflow}.

\subsection{Chosen Data Mining Algorithms}
% The Separability and Thresholds (SEaTH) algorithm \citep{Nussbaum2006}
% statistically identifies characteristic features and their thresholds. It has
% been used on remote sensing data for land cover classification \citep{Gao2011}
% and nuclear installation classification \citep{Nussbaum2006}. 
%Das muss in die Einleitung
Using training data, the algorithm determines the separability of the object classes and then
calculates the thresholds for which the maximum separability can be achieved
using the given features based on the 
%%%
\ldots. distance...
One benefit of the algorithm is that one does not need many training objects. In
\cite{Nussbaum2006}, for example, the authors suggest using only very
characteristic features for training and only used around 10 samples per
class. The authors also state that usually two features
per class is enough to produce accurate results.\\
%% DT
The used DT implementation is a modified classification and
regression tree (CART)\citep{scikit-learn}. We use an evolutionary search 
algorithm with 10 evolutions based on the Distributed Evolutionary Algorithms 
in Python \citep{DEAP_JMLR2012} software to find the best features. The 
DT fitted with the features chosen by the evolutionary search algorithm 
are then compared with a DT trained on all features and 
a DT trained on the top 10 most important features as chosen by the Extra 
Trees (ET) classifier. A 5 fold cross validation is performed and rules from 
the highest overall accuracy DT is chosen. The ET classifier is a modified 
random forest without bootstrapping and instead of splitting on the most 
discriminative feature, random thresholds from candidate features are taken.
In this work ET serves as a reference (benchmark) classifier since it it showed
very good results on several recent classification problems. 
%%
\subsection{Segmentation} 
\label{subsec_segmentation}
The iterative object-based image analysis is performed by \cite{Tintrup2015}
using Defiens eCognition Server and segments the data using thresholds and multi-resolution approaches 
\citep{baatz2001ecognition}.The data is segmented by
multi-spectral (B, G, R, NIR) orthophotos with a 0.2m ground resolution and a 2
x 2km tile size. Spectral information and indices (Bare Area Index) and
height from the DEM to separate between biotic and
abiotic features \citep{Tintrup2015}. The detailed pre-processing workflow is
shown in figure \ref{fig_pre-processing}.

\section{Data and Study Area}
Saarburg is an 200km$^{2}$ administrative district and is located in the
south-west of the federal state of Rhineland Palatinate (RLP), Germany.
Luxembourg borders the area to the west and the federal state of Saarland to
the South. RLP has a western european atlantic climate and has an economically
and culturally important viticulture industry along the Mosel and Rhine rivers.
\begin{figure}
\label{fig_study_area}
    \includegraphics[width=\textwidth]{diagrams/study_area_closeup.png}
    \caption{The location of Saarburg on the left in purple in relation to
    Rheinland Palatinate. Map on right \copyright Thunderforest, Data
\copyright OpenStreetMap contributors.}
\end{figure}
The digital orthophotos and are used to create the Digital Surface Model (DSM)
using automated stereo matching and therefore represent the basis for all
segmentation and classification procedures.
Available orthophotos are from the years 2xxx, 2xxx, 2013.
The DTM and DEM was produced using LiDAR ASCII point clouds acquired between 2003 and 2009
with a resolution of 0.5m.\\
\subsection{Thematic Maps}
The RLP biotope map was last updated in 2015. The agriculture data set (INVEKOS) is updated yearly. 
% The Forestry map is from 200X and the soil map is from 200X.

\begin{figure} \includegraphics[width=1\textwidth]{diagrams/pre_processing.png}
    \caption{Biotope, Agriculture, Forestry, Geology (soil maps) are unioned to
    then be joined by the segmented polygons derived from orthophotos. The
    features are joined together and get all properties associated from the
    dataset unless they conflict. When conflicts occur the corresponding column
    is marked as such.}
\label{fig_pre-processing}
\end{figure}
%Die Datenaufbereitung muss ebenfalls noch besser beschrieben werden.

\subsection{Validation} 
The data was divided into 20\% for training and 80\% for validation. For SEaTH
20 objects per class were used from the training data set because the algorithm
performs better with a smaller number of features \citep{Nussbaum2006}. We 
selected only objects that were identified to be herbaceous plants by the 
segmentation alorithm for testing and training. The testing 
data includes polygons of grasslands between trees in orchards and 
algricultural land which are grasslands. %To evaluate the quality of the results
%in respect to well-established, popular classification approaches the outcomes
% were compared to a ET reference classification (see table \ref{xxx})

\section{Results}
Training the data with the ET classifier yielded classification results of
100\% for indicators usage, wetness and

Mesic grasslands represented the most frequent grassland class in the Saarburg 
region. The classification report and results from training the DT algorithm  
shows that it is well represented in the data.
\begin{tabular}{c c c c c}
Class & Precision & Recall & F-score & Support\\
\hline\\
D5 & 0.0 & 0.0 & 0.0 & 1\\
E1 & 0.0 & 0.0 & 0.0 & 1\\
E2 & 0.75 & 1.0 & 0.86 & 10409\\
E3 & 0.0 & 0.0 & 0.0 & 3\\
F3 & 0.0 & 0.0 & 0.0 & 3\\
FB & 0.0 & 0.0 & 0.0 & 460\\
G1 & 0.0 & 0.0 & 0.0 & 25\\
H2 & 0.0 & 0.0 & 0.0 & 1\\
I1 & 0.0 & 0.0 & 0.0 & 2922\\
J1 & 0.0 & 0.0 & 0.0 & 9\\
avg & 0.57 & 0.75 & 0.65 & 13834\\
\label{fig_mesic_classification}
%\caption{Classification report of mesic grasslands from scikit-learn}
\end{tabular}

The few dry grasslands in the Saarburg region really hurt the classifier's 
ability to learn from the data. Only one dry grassland classified as such fits 
the rule's definition of a dry grassland. The reasons for this could be due to 
the differences in data quality between the biotope map and the agriculture map 
with grasslands misclassified or a conceptualization problem with the 
translation from the RLP classification to EUNIS. 

Visually inspecting the data revealed that polygons were in fact grasslands, 
agricultural land or meadow as the height restriction and the class name 
``herbaceous plants'' from the segmentation properly excluded trees, artificial 
buildings, etc. The larger size of the thematic map's 
polygons means that a polygon of grasslands in between an orchard will still be 
assigned the orchard's label and EUNIS indicators. Early on the issue with
orchards was identified and even after having moved to a minimum 200m$^{2}$ size for
polygons created during segmentation has not been resolved. A multi-scale
approach will probably be needed to solve this problem.
%% overal 

\begin{tabular}{c c c c c}
Class & Precision & Recall & F-score & Support\\
\hline\\
D5 & 0.0 & 0.0 & 0.0 & 0\\
E1 & 0.01 & 1.0 & 0.01 & 1\\
E2 & 0.0 & 0.0 & 0.0 & 103\\
E3 & 0.0 & 0.0 & 0.0 & 0\\
F3 & 0.0 & 0.0 & 0.0 & 0\\
FB & 0.0 & 0.0 & 0.0 & 25\\
G1 & 0.0 & 0.0 & 0.0 & 0\\
H2 & 0.0 & 0.0 & 0.0 & 0\\
I1 & 0.0 & 0.0 & 0.0 & 53\\
J1 & 0.0 & 0.0 & 0.0 & 0\\
avg & 0.0 & 0.01 & 0.0 & 182\\
\label{fig_dry_classification}
%\caption{Classification report of mesic grasslands from scikit-learn}
\end{tabular}
The very wet results also reflect the few wet or very wet grasslands in the 
region.
\begin{tabular}{c c c c c}
Class & Precision & Recall & F-score & Support\\
\hline\\
D5 & 0.0 & 0.0 & 0.0 & 0\\
E1 & 0.0 & 0.0 & 0.0 & 0\\
E2 & 0.0 & 0.0 & 0.0 & 55\\
E3 & 0.01 & 1.0 & 0.02 & 1\\
F3 & 0.0 & 0.0 & 0.0 & 0\\
FB & 0.0 & 0.0 & 0.0 & 2\\
G1 & 0.0 & 0.0 & 0.0 & 0\\
H2 & 0.0 & 0.0 & 0.0 & 0\\
I1 & 0.0 & 0.0 & 0.0 & 32\\
J1 & 0.0 & 0.0 & 0.0 & 0\\
avg & 0.0 & 0.01 & 0.0 & 90\\
\label{fig_very_wet_classification}
\end{tabular}

Adding immature soil and the usage intensity increases classification accuracy 
only very slightly.

\section{Discussion}
According to our conceptualization of the EUNIS grasslands, most grasslands 
have a height of less than one meter. The normalized digital surface model's 
max height for graslands is quite a bit bigger than that.
%% insert table
The misclassification of I1 - ``Arable land and market gardens'' is 
understandable as the EUNIS biotope classification specifically states that 
turf and sports fields are excluded. Since I1 is ``species poor'' adding 
``species richness'' to the E2 rule reduced the number of polygons classified 
as I1 but also reduced the total number of E2 classes as well.\\
\\
SEaTH shows much better separability when one chooses
larger objects and fewer training objects per class. Training SEaTH on a few
carefully selected objects being ideal class representations further increases
the separability, but the classification accuracy suffers when applied to the
complete data set. Thus, SEaTH appears to over-fit. Using two sets of rules 
produced from different sized training data produced quite different results 
when tested on the same dataset. 
\\
The biotope classification is much more 
accurate than the agricultural data and only polygons where both maps agreed 
should have been selected.
\section{Conclusion}
We showed an automated workflow using ontologies and data mining algorithms can 
accurately classify EUNIS habitat objects. Moreover, the use of ontologies if 
published on the Internet can allow others to reuse the workflow and make their 
own modifications. 
\section{Acknowledgments}
We would like to thank RLP AgroScience GmbH for processing the data and the
RLP's Environment Ministry for funding the NATFLO project. This work was
conducted using the Prot\'eg\'e resource, which is supported by grant GM10331601
from the National Institute of General Medical Sciences of the United States
National
Institutes of Health.
\bibliographystyle{model2-names}
\section{References} \bibliography{references} \end{document}
