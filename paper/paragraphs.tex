
\section{State of Research and Research Question}
Remote sensing has a rich history in environmental conservation, meteorology,
climate science and other fields. Due to this history remote sensing experts
have become very good at analyzing remote sensing data. Yet, remote sensing
image analysis implicitly incorporates the expertise and knowledge of the
individual performing the analysis and reduces objectivity. This can be divided
into remote sensing knowledge (spectral signature, remote sensing index, etc.)
and field knowledge (feature properties, spatial relations, etc)
\citep{Andres2013a}. This knowledge is often neither completely nor explicitly
defined but influences the classification. Thus, two experts can interpret the
same image differently due to their unique conceptualizations and experiences.
Further complicating matters, the classification chain is not documented and
controlled, reducing comparability and hindering attempts to reproduce the
results\citep{Arvor2013}. Therefore automated methods for remote sensing
classifications that produce accurate and reproducible results are desired.
An automated system would also reduce the time needed to analyze large
remote sensing datasets. Many approaches have been used to create an accurate
automated classification tool using statistics and different algorithms
from machine learning and data mining. Unfortunately most of these approaches
rely on experts to produce rule-sets or manually select training points. The
former relies on \emph{a priori} knowledge while the latter can be
time-consuming.
Neither are observation-based. There are many different automatic methods
for feature selection. A brief overview explaining the most common methods is
described below.